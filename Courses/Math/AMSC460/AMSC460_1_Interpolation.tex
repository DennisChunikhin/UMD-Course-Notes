\documentclass[12pt,letterpaper]{article}

%\usepackage{common}
\usepackage{amsmath}
\usepackage{amssymb}
\usepackage{mathtools}
\usepackage{nicematrix}
\usepackage{algorithm}
\usepackage{algpseudocode}
\usepackage{tcolorbox}
\tcbuselibrary{theorems}

\usepackage{hyperref}
\usepackage{parskip}

\renewcommand{\algorithmicrequire}{\textbf{Input:}}
\renewcommand{\algorithmicensure}{\textbf{Output:}}

\newcommand{\skipline}{\vspace{\baselineskip}}
\newcommand{\dis}{\displaystyle}
\newcommand{\noin}{\noindent}

\newcommand{\R}{\mathbb{R}}
\newcommand{\C}{\mathbb{C}}
\newcommand{\mt}[1]{\text{#1}}
\newcommand{\tr}{\mathrm{Trace}}
\newcommand{\Tr}{\mathrm{Trace}}
\newcommand{\Ker}{\mathrm{Ker}}
\newcommand{\dd}{\mathrm{d}}
\newcommand{\fl}{\mathrm{fl}}

\DeclarePairedDelimiter\abs{\lvert}{\rvert}

\newtcbtheorem[number within=section]{theo}{Theorem}{colback=green!5,colframe=green!35!black,fonttitle=\bfseries}{th}
\newtcbtheorem[number within=section]{coro}{Corollarly}{colback=green!5,colframe=green!35!black,fonttitle=\bfseries}{th}
\newtcbtheorem[number within=section]{lemma}{Lemma}{colback=green!5,colframe=green!35!black,fonttitle=\bfseries}{th}
\newtcbtheorem[number within=section]{defn}{Definition}{colback=red!5,colframe=red!35!black,fonttitle=\bfseries}{th}
\newtcbtheorem[number within=section]{constr}{Construction}{colback=yellow!5,colframe=yellow!35!black,fonttitle=\bfseries}{th}
\newtcbtheorem[number within=section]{algo}{Algorithm}{colback=blue!5,colframe=blue!35!black,fonttitle=\bfseries}{th}
%\newtcbtheorem[number within=section]{lemma}{Lemma}{colback=yellow!5,colframe=yellow!35!black,fonttitle=\bfseries}{th}

\title{AMSC 460 Notes}
\author{Dennis Chunikhin}
\date{\today}

\begin{document}

\maketitle

\newpage
\tableofcontents
\newpage

\section{Polynomial Interpolation}

\textbf{Assume the following for this entire section:}

We have an arbitrary function $f(x)$.
\begin{itemize}
	\item We pick $n+1$ distinct points $$x_0 \neq x_1 \neq \dots \neq x_n \in \R$$
	\item We evaluate the function at these points: $$y_0 = f(x_0), y_1 = f(x_1), \dots, y_n = f(x_n)$$
	\item We want to find n-th degree polynomial $$p_n(x) = a_0 + a_1 x + \dots + a_n x^n$$
	such that the polynomial equals $f(x)$ at all $x_i$:
	\begin{align*}
		p_n(x_0)&=y_0 =f(x_0) \\ p_n(x_1) &= y_1 = f(x_1)  \\ &\ \vdots \\ p_n(x_n) &= y_n = f(x_n)
	\end{align*}
\end{itemize}

\subsection{Vandermonde Matrix}

Notice that we can write the polynomial as a product of a row and column vector:
\begin{align}
	p_n(x) = a_0 + a_1 x + a_2 x^2 + \dots + a_n x^n = \begin{pmatrix}
		1 & x & x^2 & \cdots & x^n
	\end{pmatrix} \begin{pmatrix}
		a_0 \\ a_1 \\ a_2 \\ \vdots \\ a_n
	\end{pmatrix} \label{eq:vandermonde_1}
\end{align}

We want this to hold for each $x_i$ and $y_i$, so we concatenate the row-vectors on the left hand side of equation \ref{eq:vandermonde_1} as rows in a \textbf{Vandermonde} matrix and set the product equal to a vector of the $y$'s:
\begin{defn}{Polynomial interpolation using the Vandermonde matrix}
.To find the coefficients of the interpolating polynomial $p_n(x)$, we need to solve the following linear system of equations for the coefficient vector \textbf{a}:
\begin{equation}
\begin{pmatrix}
	1 & x_0 & x_0^2 & \cdots & x_0^n \\
	1 & x_1 & x_1^2 & \cdots & x_1^n \\
	\vdots & \vdots & \vdots & \ddots & \vdots \\
	1 & x_n & x_n^2 & \cdots & x_n^n
\end{pmatrix}
\begin{pmatrix}
	a_0 \\ a_1 \\ \vdots \\ a_n
\end{pmatrix} =
\begin{pmatrix}
	y_0 \\ y_1 \\ \vdots \\ y_n
\end{pmatrix}
\end{equation}

The matrix on the left-hand side of this equation is called the \textbf{Vandermonde matrix}.
\label{def:1.1}
\end{defn}

\begin{theo}{A solution to the system exists}
.A solution to the system of equations in definition \ref{def:1.1} always exists.
\label{thm:1.1}
\end{theo}

\subsubsection*{Proof of theorem \ref{thm:1.1}}

If we can show that the Vandermonde matrix is invertible, we prove that a solution exists.

To show that the Vandermonde matrix is invertible, we need to show that its columns are linearly independent. That is, that the only solution to
\begin{equation} \label{eq:2}
c_0 \begin{pmatrix}
	1 \\ 1 \\ \vdots \\ 1
\end{pmatrix} +
c_1 \begin{pmatrix}
	x_0 \\ x_1 \\ \vdots \\ x_n^n
\end{pmatrix} +
c_2 \begin{pmatrix}
	x_0^2 \\ x_1^2 \\ \vdots \\ x_n^n
\end{pmatrix} + \dots +
c_n \begin{pmatrix}
	x_0^n \\ x_1^n \\ \vdots \\ x_n^n
\end{pmatrix}  = 0
\end{equation}
is $c_0 = c_1 = \dots = c_n = 0$.

Notice that we can write equation \ref{eq:2} as
\begin{equation} \label{eq:3}
\begin{pmatrix}
	c_0 + c_1 x_0 + c_2 x_0^2 + \dots + c_nx_0^n \\
	c_0 + c_1 x_1 + c_2 x_1^2 + \dots + c_nx_1^n \\
	\vdots \\
	c_n + c_1 x_n + c_2 x_n^2 + \dots + c_nx_n^n
\end{pmatrix} = 0
\end{equation}

Let's denote the polynomial $c_0 + c_1 x + c_2 x^2 + \dots + c_n x^n = C(x)$.

Notice that each row of equation \ref{eq:3} is the polynomial $C$ evaluated at $x_i$, and is equal to zero. This implies that each $x_i$ is a root of the polynomial $C$. However $C$ is an n-th degree polynomial, meaning it must have at most $n$ distinct roots, or be identically zero ($C(x)=0$). Since there are $n+1$ distinct $x_i$, we must have $C(x) = c_0 + c_1x + \dots + c_n x^n = 0 \implies c_0=c_1=\dots=c_n=0$.

Thus the columns are linearly independent, the Vandermonde matrix is invertible, and a solution to the linear equation in definition \ref{def:1.1} exists!

\subsection{Lagrange polynomials}

Alternatively we can interpolate using \textbf{Lagrange polynomials:}
\begin{defn}{Lagrange Polynomials}
.The k-th Lagrange polynomial is:
\[
l_k(x) = \frac{(x-x_0)(x-x_1)\dots(x-x_{k-1})(x-x_{k+1})\dots(x-x_n)}{(x_k-x_0)(x_k-x_1)\dots(x_k-x_{k-1})(x_k-x_{k+1})\dots(x_k-x_n)}
\]
\label{def:lagrange}
\end{defn}

This polynomial is constructed to be zero at all $x_i, i\neq k$ and $1$ at $x_k$. This makes it very useful for finding interpolating polynomials mathematically:
\begin{theo}{Interpolation using Lagrange polynomials}
.The interpolating polynomial is given by:
$$p_n(x) = y_0 l_0(x) + y_1 l_1(x) + \dots + y_n l_n(x)$$
\label{thm:lagrange_interpolation}
\end{theo}
This is inefficient to compute, but is mathematically useful.

\section{Accuracy of Interpolation}

Take the same setup as in section 1:

We are interpolating arbitrary function $f(x)$ on the interval $[a, b] \subseteq \R$ at $n+1$ points with an $n$-th degree polynomial $p(x)$.

We make the additional assumption that the function $f$ is defined on the interval $[a,b]$ and that $f^{(n+1)}$ is well defined on $[a,b]$.

\begin{constr}{Setup}
.Pick an arbitrary point in the interval $$\xi \in [a,b]$$

Define
\begin{equation}\label{eq:def_L}
L(x) = (x-x_0)(x-x_1) \dots (x-x_n)
\end{equation}
This is the ''simplest'' polynomial that has the roots $x_0, \dots, x_n$.

\skipline

Define
\begin{equation}
\alpha = \frac{f(\xi) - p(\xi)}{L(\xi)}
\end{equation}

\skipline

Finally, define
\begin{equation}\label{eq:def_F}
F(x) = f(x) - p(x) - \alpha L(x)
\end{equation}
All of these constructions designed so that $F$ has zeros at $x_0, \dots, x_n$ and $\xi$:
\begin{equation}\label{eq:7}
F(x_0) = F(x_1) = \dots = F(x_n) = F(\xi) = 0
\end{equation}
\end{constr}

Notice that by equation \ref{eq:7}, $F$ has $n+2$ roots in $[a,b]$.

By Rolle's theorem,
\begin{align*}
	F'(x) \text{ has } &n+1 \text{ roots} \\
	F''(x) \text{ has } &n \text{ roots} \\
	&\ \vdots \\
	F^{(n+1)}(x) \text{ has } &1 \text{ root}
\end{align*}

\begin{constr}{$\eta$}
.Let $\eta \in [a,b]$ be the root of $F^{(n+1)}$

\medskip
That is,
\begin{equation}
	F^{(n+1)}(\eta) = 0
\end{equation}
\label{constr:mu}
\end{constr}

By the definition of $F$ (equation \ref{eq:def_F}),
\begin{equation}\label{eq:poly_bound_1}
	F^{(n+1)}(\eta) = f^{(n+1)}(\eta) - p^{(n+1)}(\eta) - \alpha L^{(n+1)}(\eta) = 0
\end{equation}

Since the interpolating polynomial $p$ is an $n$-th degree polynomial, we have $p^{(n+1)} = 0$.

Thus equation \ref{eq:poly_bound_1} becomes
\begin{equation}\label{eq:poly_bound_2}
	f^{(n+1)}(\eta) - \alpha L^{(n+1)}(\eta) = 0
\end{equation}

Notice that by the definition of $L$ (equation \ref{eq:def_L}), $L$ is an $n+1$ degree polynomial with leading term $x^{(n+1)}$, so
\begin{equation}
	L^{(n+1)} = (n+1)!
\end{equation}
Thus equation \ref{eq:poly_bound_2} becomes
\begin{equation}
	f^{(n+1)}(\eta) - \alpha (n+1)! = 0
\end{equation}
which gives us
\begin{align}
	f^{(n+1)}(\eta) &= \alpha (n+1)! \\
	&= \frac{f(\xi) - p(\xi)}{L(\xi)} (n+1)!
\end{align}
This finally gives us the result
\begin{align}
	f(\xi) - p(\xi) &= \frac{f^{(n+1)}(\eta)}{(n+1)!} L(\xi) \\
	&= \frac{f^{(n+1)}(\eta)}{(n+1)!} (\xi - x_0)(\xi - x_1) \dots (\xi - x_n)
\end{align}

\begin{theo}{Polynomial interpolation error}
.The error of polynomial interpolation at any point $\xi$ is given by
\begin{equation}
	f(\xi) - p(\xi) = \frac{f^{(n+1)}(\eta)}{(n+1)!}(\xi - x_0)(\xi - x_1) \dots (\xi-x_n)
\end{equation}
\label{thm:poly_error}
\end{theo}
\begin{coro}{Polynomial interpolation error bound}
.The error of polynomial interpolation is bounded by
\begin{equation}
	\big| f - p \big| \leq \left| \frac{f^{(n+1)}(\eta)}{(n+1)!} \right| \max_{\xi \in [a,b]} \big| (\xi - x_0) \dots (\xi-x_n) \big|
\end{equation}
This is a direct consequence of theorem 2.1 (take the absolute value of the result).
\label{thm:poly_error_bound}
\end{coro}

\begin{algo}{Efficiently evaluating a polynomial}
.Evaluate a polynomial $p(x) = a_0 + a_1x + \dots + a_nx^n$ at some point $x$.

\medskip

We rewrite $p(x) = a_0 + x(a_1 + x(a_2 + \dots ))$.
\begin{algorithm}[H]
\caption{Evaluate a polynomial}
\begin{algorithmic}[1]
\State $p = a_n$
\For{$i = n-1$ to $0$}
	\State $p = a_i + xp$
\EndFor
\end{algorithmic}
\end{algorithm}
\label{alg:poly_eval}
\end{algo}

\subsection{Piecewise Interpolation}
Break interval into $n$ segments and do polynomial interpolation on each of the segments.

Let $h = \max_{1 \leq i \leq n} \big| x_i - x_{i-1} \big|$ denote the maximum width of the segments.

Then, by adding up the error for each of the intervals as given by theorem \ref{thm:poly_error_bound}, we get:
% TODO: Generalize?
\begin{theo}{Linear piecewise interpolation error}
.The error bound for linear (degree 1) piecewise interpolation $l(x)$ using $n$ intervals with maximum interval width $h$ is given by
\begin{equation}
	\big| f - l \big| \leq \max_{\eta \in [x_0, x_n]} \frac{\big| f''(\eta) \big|}{2} \frac{h^2}{4}
\end{equation}
\label{thm:piecewise_error}
\end{theo}

\textbf{Benefit:} We reduce error as we add intervals (as $h \rightarrow 0$).

\textbf{Con:} The derivative is not continuous.

\section{Least Squares Fitting}
Consider data points consisting of a function $f$ which, evaluated at points $x_0, x_1, \dots, x_n$, yields $y_0, y_1, \dots, y_n$.

We want to approximate $f$ using these datapoints with polynomial $\hat{f}$ of degree $m < n$.

As in section 1, this problem can be represented with the Vandermonde matrix
\begin{constr}{Least squares fitting problem}
.Consider fitting polynomial
\begin{equation}
	\hat{f}(x) = a_0 + a_1 x + \cdots + a_m x^m
\end{equation}

Ideally, we want $\hat{f}(x_i) = y_i$ for all $i=1, \dots, n$, which we write down as the following matrix equation
\begin{equation}
\begin{pmatrix}
	1 & x_0 & x_0^2 & \cdots & x_0^m \\
	1 & x_1 & x_1^2 & \cdots & x_1^m \\
	\vdots & \vdots & \vdots & \ddots & \vdots \\
	1 & x_n & x_n^2 & \cdots & x_n^m
\end{pmatrix}
\begin{pmatrix}
	a_0 \\ a_1 \\ \vdots \\ a_m
\end{pmatrix} =
\begin{pmatrix}
	y_0 \\ y_1 \\ \vdots \\ y_n
\end{pmatrix}
\end{equation}

We denote the big matrix $A$ and write this equation more concisely as
\begin{equation}
	Aa = y
\end{equation}
\label{constr:least_squares}
\end{constr}

Unless we get incredibly lucky, the system $Aa = y$ has no solutions. Instead we want to find coefficient vector $a$ that minimizes the error.

\begin{defn}{$l^2$ norm}
.The $l^2$ norm is the typical distance metric we are used to. It is defined for a vector $\mathbf{x}$ as
\begin{equation}
\left\lVert 
\begin{pmatrix}
	x_1 \\ \vdots \\ x_n
\end{pmatrix}
\right\lVert_2 = \sqrt{x_1^2 + \cdots + x_n^2}
\end{equation}

Note that
\begin{equation}
\mathbf{x}^T \mathbf{x} = \begin{pmatrix}
	x_1 & \cdots & x_n
\end{pmatrix} \begin{pmatrix}
	x_1 \\ \vdots \\ x_n
\end{pmatrix} = x_1^2 + \cdots + x_n^2 = \lVert \mathbf{x} \rVert_2^2
\end{equation}

Arguably this is a better (more generalizable) definition for $l^2$ norm.

\label{defn:l2_norm}
\end{defn}

We define the error using this distance metric:
\begin{equation}
	\left\lVert y - Aa \right\rVert_2
\end{equation}

We want to minimize this. Note that minimizing this is equivalent to minimizing
\begin{align}
	\lVert y - Aa \rVert_2^2 &= (y - Aa)^T (y - Aa) \\
	&= (y^T - (Aa)^T) (y - Aa) \\
	&= y^T y - (Aa)^T y - y^T (Aa) + (Aa)^T (Aa) \label{eq:norm_sq}
\end{align}

We note that for two vectors $x, y$, $x^T y = y^T x$ (this may look familiar, since $x \cdot y = x^T y$ is a common way to define the dot product, and the dot product is commutative).

We also note that $(Aa)^T = a^T A^T$

Thus equation \ref{eq:norm_sq} becomes
\begin{equation}
	y^T y - 2y^T Aa + a^T A^T A a = y^T y - 2(A^T y)^T a + a^T A^T A a
\end{equation}

To mimize this, we set its gradient (with respect to $a$) equal to $0$:
\begin{align}
	\nabla (y^T y - 2(A^T y)^T a + a^T A^T A a) &= \nabla (- 2(A^T y)^T a + a^T A^T A a) \\
	&= -2A^T y + 2A^TAa = 0 \label{eq:grad}
\end{align}
The calculation of the gradient can be tedious to show but is an approachable exercise in multivariable calculus. Thus we state it without proof.

Equation \ref{eq:grad} gives us
\begin{equation}
	A^T A a = A^T y
\end{equation}

\begin{theo}{The normal equations}
.The least squares polynomial fit for equation $Aa = y$ is given by the solution to
\begin{equation}
	A^T A a = A^T y
\end{equation}

This system of equations is called the \textbf{normal equations}.
\label{thm:normal_eqns}
\end{theo}


\section{Numerical Integration}

Given arbitrary function $f(x)$ defined on $[a,b] \subseteq \R$, we want to approximate
\begin{equation}
	I(f) = \int_a^b f(x) \dd x
\end{equation}

We do this by interpolating the function and then taking the integral of the interpolating polynomial. The specific procedure (crucially, the points used for the interpolation) is called a \textbf{quadrature rule}.

\begin{defn}{Quadrature}
.We approximate $$I(f) \approx Q(f)$$
using a \textbf{quadrature rule} $Q$, in general given by
\begin{equation}\label{eq:quadrature}
	Q(f) = (b-a) \sum_{k=1}^m w_k f(x_k)
\end{equation}
for some weights $w_k$ and some arbitrarily chosen points $x_k$ at which we evaluate the function $f$.
\label{def:quadrature}
\end{defn}
The weights come about by integration as we will soon see.

\subsection{General Quadrature Rule}

We interpolate at points $x_0, x_1, \dots, x_k$ with $k$-th degree interpolating $p_k(x)$.

Using Lagrange polynomials, we have:
\begin{equation}
	p_k(x) = \sum_{i=0}^k f(x_i) l_i(x)
\end{equation}

So the quadrature rule is:
\begin{equation}
	Q(f) = \sum_{i=0}^k f(x_i) \int_a^b l_i(x) \dd x
\end{equation}

\begin{theo}{General quadrature rule}
.The general degree $k$-th degree quadrature rule is given by
\begin{equation}\label{eq:general_quad}
	Q(f) = \sum_{i=0}^k f(x_i) \int_a^b l_i(x) \dd x
\end{equation}
\label{thm:general_quad}
\end{theo}

\begin{defn}{Newton-Cotes rules}
.If the points of interpolation $x_0, \dots, x_k$ are uniformly spaced, the quadrature rules are called \textbf{Newton-Cotes rules}.
\label{def:Newton_Cotes}
\end{defn}

\subsection{Rectangle Rule}
Take a $0$-degree interpolant $p_0(x) = f(a)$ of $y_0 = f(a)$ at $x_0 = a$.

The quadrature rule is given by
\begin{equation}
	Q_0(f) = \int_a^b p_0(x) \dd x = \int_a^b f(a) \dd x = (b-a) f(a)
\end{equation}

\begin{defn}{Rectangle rule}
.The rectangle rule is the $0$-th degree Newton-Cotes method interpolated at the point $x_0=a$.

\medskip

The quadrature rule is given by
\begin{equation}
	Q_0(x) = (b-a)f(a)
\end{equation}
\label{defn:rect_rule}
\end{defn}

\subsection{Trapezoid Rule}
Take a degree $1$ Newton-Cotes interpolant $p_1(x)$ interpolating $f$ at points $x_0 = a$ and $x_1 = b$.

$p_1(x)$ is a line which intersects $f$ at $x=a$ and $x=b$. Thus,
\begin{equation}
	p_1(x) = \frac{f(b) - f(a)}{b-a}(x - a) + f(a)
\end{equation}

Therefore the quadrature rule is given by
\begin{align}
	Q_1(f) &= \int_a^b p_1(x) \dd x \\
	&= \frac{f(b) - f(a)}{2(b-a)}(x-a)^2 + f(a)x \bigg|_a^b \\
	&= (b-a) \frac{f(a) + f(b)}{2}
\end{align}

\begin{defn}{Trapezoid rule}
.The trapezoid rule is the $1$-st degree Newton-Cotes method interpolated at the points $x_0 = a$ and $x_1 = b$.

\medskip

The quadrature rule is given by
\begin{equation}
	Q_1(f) = (b-a) \frac{f(a) + f(b)}{2}
\end{equation}
\label{defn:trap_rule}
\end{defn}

\subsection{Simpson's Rule}
Take a degree $2$ Newton-Cotes interpolant $p_2(x)$ interpolating $f$ at the points $x_0 = a$, $x_1 = \frac{a+b}{2}$, and $x_2 = b$.

We can follow the general quadrature rule procedure, wherein we first write down $p_2(x)$ using lagrange polynomials:
\begin{align}
	p_2(x) &= f(a) l_0(x) + f(\frac{a+b}{2}) l_1(x) + f(b) l_2(x) \\
	&= f(a) \frac{(x-\frac{a+b}{2})(x-b)}{(a-\frac{a+b}{2})(a-b)} \\
	&\quad + f(\frac{a+b}{2})\frac{(x-a)(x-b)}{(\frac{a+b}{2}-a)(\frac{a+b}{2}-b)} \\
	&\quad + f(b) \frac{(x-a)(x-\frac{a+b}{2})}{(b-a)(b-\frac{a+b}{2})}
\end{align}

The quadrature rule is given
\begin{equation}
	Q_2(x) = \int_a^b p_2(x) \dd x = f(a) \frac{1}{6} + f\left(\frac{a+b}{2}\right) \frac{2}{3} + f(b) \frac{1}{6}
\end{equation}

The process of simplification and integration of $p_2(x)$ is tedious to typeset so it is omitted, but is in general not too difficult to do.

\begin{defn}{Simpson's rule}
.Simpson's rule is the $2$-nd degree Newton-Cotes method interpolated at the points $x_0=a$, $x_1=\frac{a+b}{2}$, $x_2=b$.

\medskip

The quadrature rule is given by
\begin{equation}
	Q_2(f) = f(a) \frac{1}{6} + f\left(\frac{a+b}{2}\right) \frac{2}{3} + f(b) \frac{1}{6}
\end{equation}
\label{defn:simpson_rule}
\end{defn}

\subsection{Accuracy of Trapezoid Rule}

The trapezoid rule uses piecewise linear interpolation. We know from theorem 2.1 that the error in piecewise linear interpolation is given by
\begin{equation}
	f(x) - p_1(x) = \frac{f''(\eta_x)}{2} (x-a)(x-b)
\end{equation}

The quadrature error is then
\begin{align}
	\big| \text{Error} \big| &= \left| \int_a^b f(x) - p_1(x) \dd x \right| \\
	&= \left| \int_a^b \frac{f''(\eta_x)}{2}(x-a)(x-b) \dd x \right| \\
	&\leq \int_a^b \left| \frac{f''(\eta_x)}{2} (x-a)(x-b) \right| \dd x \\
	&\leq \max_{\eta \in [a, b]} \frac{\left| f''(\eta) \right|}{2} \int_a^b \left| (x-a)(x-b) \right| \dd x \label{eq:40}
\end{align}

Notice that the polynomial $(x-a)(x-b)$ is negative on the interval $x \in [a,b]$. Therefore equation \ref{eq:40} becomes
\begin{align}
	\big| \text{Error} \big| &\leq -\max_{\eta\in[a,b]}\frac{\left| f''(\eta) \right|}{2} \int_a^b (x-a)(x-b) \dd x \label{eq:42}
\end{align}

We integrate equation \ref{eq:42} by parts to get
\begin{align}
	\big| \text{Error} \big| &\leq -\max_{\eta\in[a,b]}\frac{\left| f''(\eta) \right|}{2}\left[\frac{1}{2}(x-a)^2(x-b) \bigg|_a^b - \frac{1}{2} \int_a^b (x-a)^2 \dd x \right] \\
	&= \max_{\eta\in[a,b]}\frac{\left| f''(\eta) \right|}{2} \frac{1}{6} (x-a)^3 \bigg|_a^b \\
	&= \max_{\eta\in[a,b]}\frac{\left| f''(\eta) \right|}{12} (b-a)^3
\end{align}

\begin{theo}{Trapezoid rule error bound}
.The Quadrature error of the trapezoid rule is bounded by
\begin{equation}
	\max_{\eta \in [a,b]} \frac{\left| f''(\eta) \right|}{12} (b-a)^3
\end{equation}
\label{thm:trap_error}
\end{theo}

\subsection{More on General Quadrature}

\begin{theo}{Quadrature of even degree perfectly integrates polynomials of 1 degree higher}
.If $k$ is even and $f$ is a polynomial of degree $k+1$, then the $k$-th degree quadrature gives the exact integral:
\begin{equation}
	Q_k(x) = I(x)
\end{equation}
\end{theo}

\subsubsection*{Sketch of proof of theorem 4.3 for the Simpson's rule}

We note that $I$ and $Q$ are linear (it is a good exercise to verify this).

Consider $f = \alpha_3 x^3 + \alpha_2 x^2 + \alpha_1 x + \alpha_0$.

\begin{align}
	I(f) &= \alpha_3 I(x^3) + I(\alpha_2 x^2 + \alpha_1 x + \alpha_0) \\
	&= \alpha_3 I(x^3) + Q(\alpha_2 x^2 + \alpha_1 x + \alpha_0)
\end{align}

and we can verify by integrating $f$ and computing the quadrature rule that $I(x^3) = Q(x^3)$.

Thus we get $I(x) = Q(x)$.


\section{Piecewise Quadrature Rules}

We split the interval $[a,b]$ into sub-intervals and use a quadrature rule on each sub-interval and adding the results up.

Let's say we split the interval at points $\{x_i\}$ into sub-intervals $[x_i, x_{i+1}]$.

\subsection{Composite Trapezoidal Rule}

Let us apply the trapezoidal quadrature rule $Q_i(f)$ to $f$ on each sub-interval $[x_i, x_{i+1}]$.

Call $I_i(f)$ the true integral of $f$ on each sub-interval $[x_i, x_{i+1}]$, that is
\begin{equation}
	I_i(f) = \int_{x_i}^{x_{i+1}} f(x) \dd x
\end{equation}

Since we can see that 
\begin{equation}
	I(f) = \sum_{i=1}^n I_i(f)
\end{equation}
the composite trapezoidal quadrature rule is given by
\begin{equation}
	Q(f) = \sum_{i=1}^n Q_i(f)
\end{equation}

\subsection{Piecewise Quadrature Error Bounds}
\subsubsection{Composite Trapezoidal Rule}
We will now estimate the error bound for the composite trapezoidal rule.

\begin{align}
	\big| \text{Error} \big| &= \left	| I(f) - Q(f) \right| \\
	&= \left| \sum_{i=1}^n \big(I_i(f) - Q_i(f) \big) \right| \\
	&\leq \sum_{i=1}^n \left| I_i(f) - Q_i(f) \right| \label{eq:53}
\end{align}

We substitute the error bound for the trapezoidal quadrature rule (theorem 4.2) into equation \ref{eq:53} to get:
\begin{align}
	\big| \text{Error} \big| &\leq \sum_{i=1}^n \max_{\eta \in [x_i, x_{i+1}]} \frac{\left| f''(\eta) \right|}{12} (x_i - x_{i-1})^3 \\
	&\leq \max_{\eta \in [a,b]} \frac{\left| f''(\eta) \right|}{12} \sum_{i=1}^n (x_i - x_{i-1})^3 \label{eq:55}
\end{align}

Let's now assume that all of the sub-intervals are of size $h$, that is $\forall x_{i+1}, x_i = h$.

Note that since there are $n$ intervals, $h = \frac{b-a}{n}$

Equation \ref{eq:55} then becomes:
\begin{align}
	\big| \text{Error} \big| &\leq \max_{\eta \in [a,b]} \frac{\left| f''(\eta) \right|}{12} \sum_{i=1}^n h^3 \\
	&\leq \max_{\eta \in [a,b]} \frac{\left| f''(\eta) \right|}{12} n h^3 \\
	&= \max_{\eta \in [a,b]} \frac{\left| f''(\eta) \right|}{12} (b-a) h^2
\end{align}

\begin{theo}{Piecewise trapezoidal quadrature rule error}
.Given sub-intervals of equal width $h$, the error in the piecewise trapezoidal quadrature rule is bounded by
\begin{equation}
	\max_{\eta \in [a,b]} \frac{\left| f''(\eta) \right|}{12} (b-a) h^2
\end{equation}
\label{thm:piecewise_trap_error}
\end{theo}

Error decreases as we decrease interval width $h$. Specifically, the error is $O(h^2)$.

\subsubsection{Composite Simpson's Rule}

The same analysis can be performed for piecewise Simpson's rule, though it is tedious and therefore omitted.

\begin{theo}{Piecewise Simpson's quadrature rule error}
.Given sub-intervals of equal width $h$, the error in the piecewise Simpson's quadrature rule is bounded by
\begin{align}
	%\max_{\xi \in [a,b]} \frac{\left| f^{(4)}(\xi) \right|}{90} \left( \frac{b-a}{2} \right)^5 \\
	\max_{\xi \in [a,b]} \frac{\left| f^{(4)}(\xi) \right|}{2880} (b-a) h^4
\end{align}
\label{thm:piecwise_simpson_error}
\end{theo}

For composite Simpson's rule, the error reduces as $O(h^4)$!

\subsection{Estimating Error in  Piecewise Quadrature}

Consider composite quadrature on interval $[\alpha, \beta]$

\subsubsection{Composite Trapezoidal Rule}
We will estimate the error bound of the trapezoid rule by performing it on different sub-intervals.

Let $Q_1$ be the composite trapezoidal quadrature rule on $n$ sub-intervals of width $h$.

Let $Q_2$ be the same quadrature rule but on $2n$ sub-intervals of width $\frac{h}{2}$.

The error bounds are
\begin{align}
	\left| I(f) - Q_1(f) \right| &\leq \epsilon_1 = \frac{\beta-\alpha}{12} \max_{\xi \in [\alpha, \beta]} \big| f''(\xi) \big| h^2 \\
	\left| I(f) - Q_2(f) \right| &\leq \epsilon_2 = \frac{\beta-\alpha}{12} \max_{\xi \in [\alpha, \beta]} \big| f''(\xi) \big| \left( \frac{h}{2} \right)^2
\end{align}

Notice that
\begin{equation}
	\epsilon_2 = \frac{1}{4} \epsilon_1
\end{equation}

That is,
\begin{align}
	& ( I - Q_1 ) \approx 4 ( I - Q_2 ) \\
	\implies & ( I - Q_1 ) - ( I - Q_2) \approx 3 (I - Q_2) \\
	\implies & (I - Q_2) \approx \frac{1}{3} (Q_2 - Q_1)
\end{align}

\begin{theo}{Composite trapezoidal rule error estimate}
.Let $Q_1$ be the composite trapezoidal quadrature rule on $n$ sub-intervals of width $h$.

Let $Q_2$ be the same quadrature rule but on $2n$ sub-intervals of width $\frac{h}{2}$.

\medskip

The error bound of $Q_2$ is approximately
\begin{equation}
	(I - Q_2) \approx \frac{1}{3} (Q_2 - Q_1)
\end{equation}
\label{thm:comp_trap_error_est}
\end{theo}

\subsubsection{Composite Simpson's Rule}
The same analysis can be performed for the composite Simpson's quadrature rule.

Let $Q_1$ be the composite Simpson's quadrature rule on $n$ sub-intervals of width $h$.

Let $Q_2$ be the same quadrature rule but on $2n$ sub-intervals of width $\frac{h}{2}$.

The error bounds are
\begin{align}
	\left| I(f) - Q_1(f) \right| &\leq \epsilon_1 = (\beta - \alpha) \max_{\xi \in [\alpha, \beta]} \frac{\big| f''(\xi) \big|}{2880} h^4 \\
	\left| I(f) - Q_2(f) \right| &\leq \epsilon_2 = (\beta - \alpha) \max_{\xi \in [\alpha, \beta]} \frac{\big| f''(\xi) \big|}{2880} \left( \frac{h}{2} \right)^4
\end{align}

Notice that
\begin{equation}
	\epsilon_2 = \frac{1}{16} \epsilon_1
\end{equation}

That is,
\begin{align}
	& ( I - Q_1 ) \approx 16 ( I - Q_2 ) \\
	\implies & ( I - Q_1 ) - ( I - Q_2) \approx 15 (I - Q_2) \\
	\implies & (I - Q_2) \approx \frac{1}{15} (Q_2 - Q_1)
\end{align}

\begin{theo}{Composite Simpson's rule error estimate}
.Let $Q_1$ be the composite Simpson's quadrature rule on $n$ sub-intervals of width $h$.

Let $Q_2$ be the same quadrature rule but on $2n$ sub-intervals of width $\frac{h}{2}$.

\medskip

The error bound of $Q_2$ is approximately
\begin{equation}
	(I - Q_2) \approx \frac{1}{15} (Q_2 - Q_1)
\end{equation}
\label{thm:comp_Simpson_error_est}
\end{theo}

\subsection{The Adaptive Quadrature Algorithm}
We can use these estimated error bounds to create an adaptive algorithm that computes a quadrature rule to an arbitrary given tolerance.

\begin{algo}{Adaptive Quadrature}
.A recursive algorithm for performing quadrature of given function $f$ on interval $[\alpha, \beta]$ to an arbitrary error tolerance $\tau$.

\begin{algorithm}[H]
\caption{Adaptive Quadrature}
\begin{algorithmic}[1]
\Require{$\alpha, \beta, f, \tau$}
\Ensure{$Q$}
\Statex
\Function{adaptq}{$\alpha, \beta, f, \tau$}
\State Compute $Q_1$ \Comment{Intervals of width $h$}
\State Compute $Q_2$ \Comment{Intervals of width $h/2$}
\item[]
\State Compute $\epsilon = \frac{1}{3} (Q_2 - Q_1)$ \Comment{for trapezoidal rule}
\State \textbf{or}
\State Compute $\epsilon = \frac{1}{15} (Q_2 - Q_1)$ \Comment{for Simpson's rule}
\item[]
\If{$\epsilon \leq \tau$}
\State $Q = Q_1$
\Else
\State $Q = $ \Call{adaptq}{$\alpha, \frac{\alpha+\beta}{2}, f, \frac{\tau}{2}$} $ + $ \Call{adaptq}{$\frac{\alpha+\beta}{2}, \beta, f, \frac{\tau}{2}$}
\EndIf
\item[]
\State \Return $Q$
\EndFunction
\end{algorithmic}
\end{algorithm}
\label{alg:adaptq}
\end{algo}

\section{Gaussian Quadrature}

Recall that by theorem 4.3, a Newton-Cotes quadrature rule of degreen $m-1$ (using $m$ points) produces the exact integral of a degree $m$ polynomial.

By selecting points to interpolate more cleverly, we can do better.

\subsection{An Orthonormal Basis for Polynomials}
We must first discuss the vector space of degree $m$ polynomials.

\begin{defn}{Polynomial inner product}
.Define the inner product of two functions $f(x)$ and $g(x)$ on interval $[a,b]$ by
\begin{equation}
	\langle f, g \rangle = \int_a^b f(x) g(x) \dd x
\end{equation}
\label{defn:poly_inner_prod}
\end{defn}

\begin{defn}{Orthogonality of polynomials}
.We say that two polynomials $p(x)$ and $q(x)$ are orthogonal if
\begin{equation}
	\langle p, q \rangle = \int_a^b pq \dd x = 0
\end{equation}
\label{defn:orthogonal}
\end{defn}

Consider the vector space of polynomials of degree $m$ or less (often denoted $\mathcal{P}_m(\R)$).

We begin with a standard basis of this vector space:
\begin{equation}
	\{1, x, x^2, \dots, x^m\}
\end{equation}

Then use the Gram-Schmidt process to create an orthonormal basis
\begin{equation}
	\{\psi_0, \dots, \psi_m\}
\end{equation}

Where each $\psi_i$ is an $i$-th degree polynomial (note that if our interval is $[-1, 1]$, these are called the Legendre polynomials).

Consider a $k$-th degree polynomial $q$. Note that since $\psi_0, \dots, \psi_m$ form a basis, we can write
\begin{equation}
	q = c_0 \psi_0 + \cdots + c_m \psi_m
\end{equation}
for some constants $c_0, \dots, c_m \in \R$.

Since for all $i > k$, $\psi_i$ is a polynomial of order greater than $k$, meaning that $\forall i > k, c_i = 0$.

Thus $q$ is a linear combination of $\psi_0, \dots, \psi_k$, meaning that
\begin{equation}
	\forall j > k, \langle \psi_j, q \rangle = 0
\end{equation}

\subsection{Zeros of the Inner Product}
\begin{lemma}{The zeros of $\psi_m$ lie in $(a, b)$}
.All roots of $\psi_m$ are real and lie in $(a,b)$.

\label{thm:zero_of_inner}
\end{lemma}

\subsubsection*{Proof of lemma 6.1}
Suppose $\psi_m$ has $k$ roots in $(a, b)$.

Denote these roots $x_1, \dots, x_k$.

Suppose $k < m$.

Let $q(x) = (x-x_1)(x-x_2) \cdots (x-x_k)$.

$q$ is a degree $k$ polynomial, and since by assumption $k < m$, $\psi_m$ is orthogonal to $q$. That is,
\begin{equation}\label{eq:psi_q_ortho}
	\int_a^b \psi_m(x) q(x) \dd x = 0
\end{equation}

Next we use a neat parity argument. Note that since $x_1, \dots, x_k$ are the only roots of $\psi_m$ on $(a,b)$, $\psi_m$ changes sign at and only these points in $(a,b)$. Note that these are also the only roots of $q$, so $q$ also changes sign at and only at these points. Because of this, we can write
\begin{equation}
	\psi_m(x) = q(x) s(x)
\end{equation}
where $s(x)$ does not change sign in $(a,b)$.

Consider
\begin{align}
	\langle \psi_m(x), q(x) \rangle &= \int_a^b \psi_m(x) q(x) \dd x \\
	&= \int_a^b q(x)^2 s(x) \dd x
\end{align}

Since neither $q(x)^2$ nor $s(x)$ change sign on $(a,b)$, and since neither are identically zero,
\begin{equation}
	\langle \psi_m(x), q(x) \rangle = \int_a^b q(x)^2 s(x) \dd x \neq 0
\end{equation}
but this is in contradiction with equation \ref{eq:psi_q_ortho}.

Thus, we must have $k = m$. That is, $\psi_m$ must have all of its $m$ roots in $(a,b)$.


\subsection{Gaussian Quadrature}
\begin{defn}{Gaussian Quadrature}
.We want to interpolate $f$ at $m$ points using a polynomial of degree $m-1$.

Start with $\psi_m$ (degree $m$).

By lemma 6.1, all roots of $\psi_m$ lie in $[a, b]$.

Denote the roots of $\psi_m$ by $x_1, \dots, x_m$.

These are the points we use for interpolation.

That is, take $p_{m-1}$ the $m-1$ degree interpolating polynomial that interpolates $f$ at $x_1, \dots, x_m$.

The quadrature rule is given by
\begin{align}
	Q(f) &= \int_a^b p_{m-1}(x) \dd x \\
	&= w_1 f(x_1) + w_2 f(x_2) + \cdots + w_m f(x_m)
\end{align}
where
\begin{equation}
	w_j = \int_a^b l_j(x) \dd x
\end{equation}
\label{def:gauss_quad}
\end{defn}

\begin{theo}{Gaussian quadrature is perfect for polynomials of degree $2m-1$}
.If $f$ is a polynomial of degree $2m-1$, then $Q(f) = I(f)$
\label{thm:gauss_perfect}
\end{theo}

\subsubsection*{Proof of theorem 6.1}
Take $f$ a polynomial of degree $2m-1$.

We can write
\begin{equation}
	f(x) = \psi_m(x) g(x) + r(x)
\end{equation}
where $g$ is of degree no more than $(2m-1)-m = m-1$ and $r$ is of degreeno more than $m-1$.

Then
\begin{align}
	Q(f) &= Q(\psi_m q + r) \\
	&= Q(\psi_m q) + Q(r) \\
	&= w_1 \psi_m(x_1)q(x_1) + \cdots + w_m \psi_m(x_m)q(x_m) + Q(r) \\
	&= w_1 \cdot 0 \cdot q(x_1) + \cdots + w_m \cdot 0 \cdot q(x_m) + Q(r) \\
	&= Q(r)
\end{align}

Since $r$ is of degree no more than $m-1$, $Q(r) = I(r)$. Thus,
\begin{equation}\label{eq:quad_f_r}
	Q(f) = I(r)
\end{equation}

Now consider
\begin{equation}
	I(f) = I(\psi_m q + r) = I(\psi_m q) + I(r)
\end{equation}

By ortogonality $I(\psi_m q) = 0$, giving us
\begin{equation}\label{eq:int_f_r}
	I(f) = I(r)
\end{equation}

So by equations \ref{eq:quad_f_r} and \ref{eq:int_f_r},
\begin{equation}
	Q(f) = I(f)
\end{equation}

\begin{theo}{Perfectly interpolating a $2m-1$ degree polynomial is the best we can do}
.There is no quadrature rule $Q$ such that $Q(f) = I(f)$ for all polynomials of degree $2m$.
\label{thm:gauss_perfect}
\end{theo}

\subsubsection*{Proof of theorem 6.2}
Let $x_1, \dots, x_m$ be the $m$ distinct points at which the polynomial is interpolated.

Consider the $2m$ degree polynomial
\begin{equation}
	f(x) = (x-x_1)^2 \cdots (x-x_m)^2
\end{equation}

Since $\forall x \in \R, f(x) \geq 0$ and is not identically $0$.
\begin{equation}\label{eq:I_greater_0}
	I(f) = \int_a^b f(x) \dd x > 0
\end{equation}

Let $p_{m-1}(x)$ be the degree $m-1$ polynomial interpolating $f$ at $x_1, \dots, x_m$.
\begin{align}
	p_{m-1} &= f(x_1) l_1(x) + \cdots + f(x_m) l_m(x) \\
	&= 0 \cdot l_1(x) + \cdots + 0 \cdot l_m(x) = 0
\end{align}

Therefore
\begin{equation}
	Q(f) = \int_a^b p_{m-1} \dd x = \int_a^b 0 \dd x = 0
\end{equation}

But since by equation \ref{eq:I_greater_0}, $I(f) > 0$,
\begin{equation}
	Q(f) \neq I(f)
\end{equation}


\section{Quadrature in Higher Dimensions}
Consider as an example quadrature in 2 dimensions.

We want to approximate
\begin{equation}
	\int_a^b \int_c^d f(x,y) \dd x \dd y
\end{equation}

Define
\begin{equation}
	I_x(f) = \int_a^b f(x,y) \dd y \approx Q(x) = \sum_{j=1}^m w_j f(x, y_j)
\end{equation}

Then
\begin{equation}
	I(f) = \int_a^b I_x \dd x \approx \sum_{j=1}^m \sum_{k=1}^{m'} w_j v_k f(x_k, y_j)
\end{equation}

\begin{defn}{2 dimensional quadrature}
.To estimate
\begin{equation}
	I(f) = \int_a^b \int_c^d f(x,y) \dd x \dd y
\end{equation}
we use quadrature
\begin{equation}
	Q(f) = \sum_{j=1}^m \sum_{k=1}^{m'} w_j v_k f(x_k, y_j)
\end{equation}
\label{def:2d_quad}
\end{defn}

Any 1 dimensional quadrature rule can be used for the weights $w_j$ and $v_k$.

We can also see that this procedure generalizes to arbitrary dimensions.

\subsection{Cost of 1D Quadrature}
\begin{defn}{Problem size}
.Define the \textbf{problem size}, denoted $N$, of a quadrature rule as the number of function evaluations needed.
\label{def:prob_size}
\end{defn}

For quadrature using $m$ points in the interval, we need $m$ function evauations (1 at each point), so the problem size is $N = m$.

Thus the total cost of a quadrature rule is $O(m) = O(N)$.

We wan tot find the total cost to achieve a given error tolerace:
\begin{equation}
	\left| I(f) - Q(f) \right| \leq \tau
\end{equation}

For the composite trapezoidal rule, $\tau \propto h^2 \propto \frac{1}{m^2} = \frac{1}{N^2}$

We want
\begin{align}
	& \frac{1}{N^2} \leq \tau \\
	\implies & N \geq \frac{1}{\sqrt{\tau}}
\end{align}

Thus
\begin{equation}
	O(N) = \frac{1}{\sqrt{\tau}}
\end{equation}

\subsection{Cost of 2D Quadrature}
Consider the composite trapezoidal rule with $m$ intervals in each direction.

We need $m^2$ function evalulations, so the problem size is $N = m^2$.

The width of each interval is proportional to $h$, so the area of each box on the domain is proportional to $h^2$.

The error on the whole domain is proportional to $h^2$.

We want $h^2 \leq \tau$.

Since $N = m^2 \propto \frac{1}{h^2}$, the condition becomes
\begin{align}
	& \frac{1}{N} \leq \tau \\
	\implies & N \geq \frac{1}{\tau}
\end{align}

So the cost is
\begin{equation}
	O(N) = \frac{1}{\tau}
\end{equation}

We can see that this logic generalizes to any dimension.

\begin{theo}{Cost of composite trapezoidal quadrature in $d$ dimensions}
.The cost of quadrature in $d$ dimensions with given error tolerance $\tau$ with respect to problem size $N$ is
\begin{equation}
	O(N) = \frac{1}{\tau^{d/2}}
\end{equation}
\label{thm:quad_d_dim}
\end{theo}

\subsection{Integration via Sampling Methods}
Notice that we can think of a vector of the points we are using for quadrature as a random variable
\begin{equation}
X = \begin{pmatrix}
	x_1 \\ \vdots \\ x_n
\end{pmatrix}
\end{equation}

Notice that the sample mean/expectation value is a quadrature rule!

Let's denote the sample mean quadrature rule by $\hat{E}(f)$.

From statistics we know that that the error given $n_s$ samples is
\begin{equation}
	I(f) - \hat{E}(f) \propto \frac{1}{\sqrt{n_s}}
\end{equation}

Thereofore if we want the error to be below tolerance $\tau$, we require
\begin{equation}
	\tau \leq \sqrt{n_s}
\end{equation}

It is interesting to ask when the composite trapezoidal rule is more efficient than the statistical method. This equates to asking for what $d$ is the following inequality satisfied
\begin{align}
	& N^{2/d} \geq \sqrt{n_s} \\
	\implies & N^{4/d} \geq n_s
\end{align}

This shows us that the composite trapezoidal rule will be more efficient when $d \leq 4$.

\begin{theo}{Statistical method more efficient for high dimensions}
.The composite trapezoidal quadrature rule is more efficient than the statistical rule for dimensions $d \leq 4$. Otherwise the statistical rule is more efficient.

\skipline

The composite Simpson's quadrature rule is more efficient than the statistical rule for dimensions $d \leq 8$. Otherwise the statistical rule is more efficient.
\label{thm:sampling}
\end{theo}


\section{Gaussian Elimination}
Given a linear system of equations $Ax = b$, where $A$ is a matrix and $x, b$ are vectors, we want to compute $x$.

We can do this using Gaussian elimination (also known as row-reduction), which we represent as a series of matrix multiplications.

Given matrix
\begin{equation}
A = \begin{pmatrix}
	a_{11} & a_{12} & \cdots & a_{1n} \\
	a_{21} & a_{22} & \cdots & a_{12n} \\
	\vdots & \vdots & \ddots & \vdots \\
	a_{n1} & a_{n2} & \cdots & a_{nn}
\end{pmatrix}
\end{equation}
we represent the first step of Gaussian elimination as multiplication by the following matrix:
\begin{constr}{$L_1$}
.We construct the matrix
\begin{equation}
L_1 = \begin{pmatrix}
	1 & 0 & 0 & \cdots & 0 \\
	-a_{21}/a_{11} & 1 \\
	-a_{31}/a_{11} & & 1 \\
	\vdots & & & \ddots \\
	-a_{n1}/a_{11} & & & & 1
\end{pmatrix}
\end{equation}

We can alternatively view this as a block matrix
\begin{equation}
L_1 =
\begin{pNiceArray}{c|c}
	1 & \mathbf{0} \\
	\hline
	\mathbf{l_1} & I_{n-1}
\end{pNiceArray}
\end{equation}
where
\begin{equation}
\mathbf{l_1} = \begin{pmatrix}
	-a_{21}/a_{11} \\
	-a_{31}/a_{11} \\
	\vdots \\
	-a_{n1}/a_{11}
\end{pmatrix}
\end{equation}
and $I_{n-1}$ is the $n-1 \times n-1$ identity matrix.
\label{constr:gauss_L1}
\end{constr}

The effect of this construction is to eliminate all entries below $a_{11}$.

\begin{theo}{$L_1^{-1}$}
.We can check that the inverse of $L_1$ is
\begin{align}
L_1^{-1} &= \begin{pNiceArray}{c|c}
	1 & \mathbf{0} \\
	\hline
	-\mathbf{l_1} & I_{n-1}
\end{pNiceArray} \\
&= \begin{pmatrix}
	1 & 0 & 0 & \cdots & 0 \\
	a_{21}/a_{11} & 1 \\
	a_{31}/a_{11} & & 1 \\
	\vdots & & & \ddots \\
	a_{n1}/a_{11} & & & & 1
\end{pmatrix}
\end{align}
\end{theo}

\begin{theo}{Gaussian Elimination Step}
.The first step in Gaussian elimination yields
\begin{equation}
L_1 A = \begin{pNiceArray}{c|ccc}
	1 & a_{12} & \cdots & a_{1n} \\
	\hline
	0 & \Block{3-3}<\Large>{\hat{A_1}} \\
	\vdots &&& \\
	0 &&&
\end{pNiceArray}
\end{equation}
where $\hat{A_1}$ is some arbitrary sub-matrix resulting from the multiplication.
\label{thm:Gauss_step_1}
\end{theo}

Having done this, we can apply the same Gaussian elimination procedure to the $n-1 \times n-1$ sub-matrix $\hat{A_1}$. This corresponds to multiplication by $n-1 \times n-1$ matrix $L_2'$:
\begin{equation}
	L_2' \hat{A_1}
\end{equation}

But we want to represent all Gaussian elimination steps as multiplications performed on $A$:
\begin{equation}
	L_2 L_1 A
\end{equation}

We can get $L_2$ from $L_2'$:
\begin{constr}{$L_2$}
.Given the matrix $L_2$ from the Gaussian elimination step on $\hat{A_1}$, we have
\begin{equation}
L_2 = \begin{pNiceArray}{c|c}
	1 & \mathbf{0} \\
	\hline
	\mathbf{0} & L_2'
\end{pNiceArray}
\end{equation}
\end{constr}

This gaussian elimination procedure and the associated constructions continue recursively, each time operating on the increasingly small sub-matrix $\hat{A}$.

The end result is that $A$ will be reduced to an upper-diagonal matrix $U$.

\begin{defn}{$U$}
.Given $n\times n$ matrix $A$, the Gaussian elimination procedure (repeated elimination for each of the $n-1$ columns, excluding the last column) generates the upper-triangular matrix $U$:
\begin{equation}
	L_{n-1} L_{n-2} \cdots L_2 L_1A = U
\end{equation}
\label{defn:U}
\end{defn}

Notice that by this definition
\begin{align}
	A &= (L_{n-1} L_{n-2} \cdots L_2 L_1)^{-1} U \\
	&= L_1^{-1} L_2^{-1} \cdots L_{n-1}^{-1} U
\end{align}

\begin{defn}{$L$}
.We define
\begin{equation}
	L_1^{-1} L_2^{-1} \cdots L_{n-1}^{-1}  = L
\end{equation}
\label{defn:U}
\end{defn}

\begin{theo}{$L$ is lower-triangular}
.We can check by multiplication that $L$ is lower-triangular.

Moreover, we can see that $L$ takes the form:
\begin{equation}
\begin{pNiceArray}{ccccc}
	1 & \\
	& 1 \\
	& & \ddots \\
	& & & 1 \\
	\mathbf{l_1} & \mathbf{l_2} & \cdots & \mathbf{l_{n-1}} & 1
\end{pNiceArray} = \begin{pmatrix}
	1 & \\
	l_{21} & 1 \\
	l_{31} & l_{32} & 1 \\
	\vdots & \vdots & & \ddots \\
	l_{n1} & l_{n2} & \cdots & l_{n,n-1} & 1
\end{pmatrix}
\end{equation}
where sub-column-matrices $\mathbf{l_1}, \dots, \mathbf{l_n}$ are as defined in construction 8.1.
\label{thm:L_upper}
\end{theo}

\begin{defn}{$LU$ decomposition}
.The $LU$ decomposition of A is given by
\begin{equation}
	A = LU
\end{equation}
where $L$ and $U$ are determined by Gaussian elimination as defined above.

$L$ is a lower-triangular matrix and $U$ is an upper-triangular matrix.
\label{defn:LU}
\end{defn}

Let's again consider the linear system of equations
\begin{equation}
	Ax = b
\end{equation}

Performing Gaussian elimination, we get
\begin{equation}
	L_{n-1} \cdots L_2 L_1 A x = L_{n-1} \cdots L_2 L_1 b
\end{equation}

The right hand side contains our definition of $U$:
\begin{equation}
	U x = L_{n-1} \cdots L_2 L_1 b
\end{equation}

\begin{defn}{$y$}
.Define
\begin{equation}
	y = L_{n-1} \cdots L_2 L_1 b
\end{equation}

Equivalently,
\begin{equation}
	L_1^{-1} L_2^{-1} \cdots L_{n-1}^{-1} y = b
\end{equation}

The left hand side contains our definition of $L$:
\begin{equation}
	Ly = b
\end{equation}
\label{defn:y}
\end{defn}

Using this definition, our system of linear equations becomes
\begin{equation}
	Ux = y
\end{equation}
where
\begin{equation}
	Ly = b
\end{equation}

\begin{algo}{Solve linear system of equations}
.Given a linear system of equations $Ax = b$, we want to solve for $x$.

\begin{algorithm}[H]
\caption{Solve $Ax = b$ for $x$}
\begin{algorithmic}[1]
\State Compute $L, U$ \Comment{$LU$ decomposition of $A$}
\State Solve $Ly = b$ \Comment{Forward substitution}
\State Solve $Ux = y$ \Comment{Back substitution}
\end{algorithmic}
\end{algorithm}

The advantage of this algorithm is that when solving systems of equations involving the same matrix $A$, $L$ and $U$ can be reused.
\label{alg:gauss}
\end{algo}

\begin{algo}{Solve $Ly = b$}
.Given triangular system of equation $Ly = b$, solve for $y$

We are given
\begin{equation}
\begin{pmatrix}
	l_{11} & \\
	l_{21} & l_{22} \\
	l_{31} & l_{32} & l_{33} \\
	\vdots & \vdots & \vdots & \ddots \\
	l_{n1} & l_{n2} & l_{n3} & \cdots & l_{nn}
\end{pmatrix}
\begin{pmatrix}
	y_1 \\ y_2 \\ y_3 \\ \vdots \\ y_n
\end{pmatrix} =
\begin{pmatrix}
	b_1 \\ b_2 \\ b_3 \\ \vdots \\ b_n
\end{pmatrix}
\end{equation}

The solution is given by:
\begin{align}
	y_1 &= \frac{b_1}{l_{11}} \\
	y_2 &= \frac{b_2 - l_{21}y_1}{l_{22}}\\
	&\ \vdots \\
	y_n &= \frac{b_n - l_{n1}y_1 - l_{n2}y_2 - \cdots - l_{n,n-1}y_{n-1}}{l_{nn}}
\end{align}

The same exact procedure in the reverse order can be used to solve the upper-triagonal system of equations $Ux = y$.
\label{alg:triagonal}
\end{algo}

\subsection{Costs}
Here we will consider the computational cost of the Gaussian elimination procedure--namely the number of floating point additions and multiplications (floating point operations, also known as FLOPs) required.

When computing the first Gaussian elimination step (multiplication of $A$ by $L_1$), for each row $i$ below the 1st row, we multiply the row $i$ by $-\frac{a_{i1}}{a_{11}}$ and add the result to the row.

This means 1 division per row and $n-1$ multiplications and additions per row.

Since there are $n-1$ rows below the 1st row, this results in $n-1$ divisions and $(n-1)^2$ multiplication/additions overall.

The Gaussian elimination is then repeated for every resulting sub-matrix $\hat{A}$.

Thus the overall number of divisions is
\begin{equation}
	\sum_{k=1}^{n-1} k
\end{equation}
which is $O(n^2)$.

The overall number of multiplications is
\begin{equation}
	\sum_{k=1}^{n-1}k^2
\end{equation}

This sum has a closed form solution that can be computed with the use of generating functions. It can also be bounded above and below by a Riemann integral approximation. Either way, we get that the sum is $\Theta\left(\frac{n^3}{3}\right)$.

Finally, having found $L$ and $U$, to solve $Ly = b$ and $Ux = y$ using algorithm 8.2, we need
\begin{equation}
	\sum_{k=1}^{n-1} k
\end{equation}
FLOPs each, which is $O(n^2)$.

\begin{theo}{Cost of Gaussian elimination}
.Given $n \times n$ matrix $A$, to solve the system of equations $Ax = b$ using Gaussian elimination requires $\Theta\left(\frac{n^3}{3}\right)$ FLOPs.

\label{thm:gauss_cost}
\end{theo}

\subsection{Floating Point Error}

\begin{defn}{Machine precision}
.Given a real number $a$, in machine computation it is approximated by $\fl(a)$ which has the property
\begin{equation}
	\frac{\left| a - \fl(a) \right|}{\left|a\right|} \leq \mu
\end{equation}
where $\mu$ is the \textbf{machine precision}, also called the \textbf{machine epsilon}.

\label{defn:machine_precision}
\end{defn}

\begin{coro}{Addition of numbers below machine precision has no effect}
.For any $\epsilon$ such that $\left| \epsilon \right| < \mu$,
\begin{equation}
	\fl(1 + \epsilon) = 1
\end{equation}
\end{coro}

\begin{coro}{$\mu$ is the smallest number whose addition has an effect}
.The machine precision $\mu$ is the smallest number such that
\begin{equation}
	\fl(1 + \mu) > 1
\end{equation}
\end{coro}

\begin{coro}{Addition to large numbers has no effect}
.For any $\epsilon$ such that $\left| \epsilon \right| < \mu$,
\begin{equation}
	\fl\left(\frac{1}{\epsilon} \pm 1\right) = \frac{1}{\epsilon}
\end{equation}
\end{coro}

If the first entry in the process of Gaussian elimination is smaller in absolute value than the machine precision, floating point error can blow up significantly.

\end{document}